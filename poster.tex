\documentclass[sigconf]{acmart}


\hypersetup{%
  pdftitle = {Cybersecurity Shuffle: Using Card Magic to Introduce
  Cybersecurity Concepts},
  pdfkeywords = {},
  pdfauthor = { Author One, Author Two},
  bookmarksnumbered,
  bookmarksopen=true,
  colorlinks=true,
  urlcolor=[rgb]{.35,0,0},
  linkcolor=[rgb]{.35,0,0},
  citecolor=[rgb]{.35,0,0},
  pdfstartview={FitH},
}

% Remove reference format block for poster proposal
\settopmatter{printacmref=false}

\setcopyright{acmcopyright}
\copyrightyear{2022}
\acmYear{2022}
\acmDOI{XX.XXXX/XXXXXXX.XXXXXXX}

%% These commands are for a PROCEEDINGS abstract or paper.
\acmConference[SIGCSE TS '22]{SIGCSE TS '22: ACM Technical Symposium on Computer Science Education}{March 02--05, 2022}{Providence, RI}
\acmBooktitle{SIGCSE TS '22: ACM Technical Symposium in Computer Science Education,
  March 02--05, 2022, Providence, RI}
\acmPrice{15.00}
\acmISBN{XXX-X-XXXX-XXXX-X/18/06}




\begin{document}

\title{Cybersecurity Shuffle: Using Card Magic to Introduce Cybersecurity
Concepts}


\author{
  Author One
}
\email{authorone@example.com}
\affiliation{
  \institution{Anonymous Organization}
  \city{Anonymous City}
  \state{Anonymous state}
  \country{Anonymous Country}
}

\author{
  Author Two
}
\email{authortwo@example.com}
\affiliation{
  \institution{Anonymous Organization}
  \city{Anonymous City}
  \state{Anonymous state}
  \country{Anonymous Country}
}


\begin{abstract}

One of the main challenges
in designing lessons
for an introductory
information security class
is how to present new technical concepts
in a manner comprehensible to students
with widely different backgrounds.
In situations like these, a non-traditional approach
can help students
engage with the material
and, in doing so,
master
these unfamiliar ideas.
We have devised a series of lessons that teach important information security
topics, such as social engineering,
side-channel attacks,
and
attacks on randomness
using card magic.
Each lesson
centers around a card trick
that illustrates how the attack works
in a manner that incorporates
student participation.
In this work,
we describe our experience using these lessons in teaching
cybersecurity topics
to high school students with limited computer science backgrounds.
Students were assessed
before and after the demonstration
to gauge their mastery of the material,
and their
opinions on each lesson.
In addition to very positive student responses
during the lesson,
our pre- and post-tests
showed that student scores for each topic improved by between 15\% and 30\%
after attending our demonstration.
Students reported that they enjoyed the lesson,
found card magic to be a useful tool,
and felt the demonstration
improved their mastery of the material.

\end{abstract}

%Hand-waving away the technical background behind these concepts is bad...
%
%...and undermines student's motivation to learn and understand the material.
%
%Bypassing the need for technical explanation while preserving the main
%  focus of the lesson.


\begin{CCSXML}
<ccs2012>
<concept>
<concept_id>10003456.10003457.10003527.10003531.10003533</concept_id>
<concept_desc>Social and professional topics~Computer science education</concept_desc>
<concept_significance>500</concept_significance>
</concept>
<concept>
<concept_id>10003456.10003457.10003527.10003541</concept_id>
<concept_desc>Social and professional topics~K-12 education</concept_desc>
<concept_significance>500</concept_significance>
</concept>
</ccs2012>
\end{CCSXML}

\ccsdesc[500]{Social and professional topics~Computer science education}
\ccsdesc[500]{Social and professional topics~K-12 education}

\keywords{Cybersecurity, Cybersecurity Education, Scaffolding, Card Magic}

\maketitle

\section{Introduction / Problem}

When teaching introductory cybersecurity classes,
it is challenging to explain
the impact of attacks and the defenses against them to
students of varying experience levels and educational backgrounds.
To make this material more approachable,
we developed three card-magic-based lessons
designed to explain
social engineering, side channel attacks, and attacks on
randomness.
Our lessons
emphasize the primary
features of the concept being taught
and offer opportunities for student interaction.
None of the tricks require advanced sleight of hand
or card manipulation,
making them easy for any instructor to quickly learn and use.

\section{Background / Related Work}
The guiding principle of our lessons is scaffolding,
a technique that uses the familiar as a bridge to
material that may be beyond a student's current reach~\cite{wood1976role}.
Our inspiration comes from
card tricks that have been successfully employed in scaffolding other
computer science concepts~\cite{bell2009computer, csunplugged,
garcia2016demystifying, curzon2008engaging}.
%Garcia et al. have produced three papers describing
%a wide variety of magic tricks, along with the computer science concepts they
%can help teach~\cite{garcia2012demystifying,
%garcia2013demystifying,
%garcia2016demystifying}.
%Similarly, Curzon et al. found success explaining computer
%science concepts to younger students using magic shows~\cite{curzon2008engaging}.


\section{Overview / Methods / Results}

We adapted our lessons into a 90 minute Zoom session
and offered it as an optional class during
a remote-learning computer science summer camp for high school
students.
The students were
asked to complete an assessment once before the lesson
and once afterwards lesson in order to measure improvement.
Following our session, aggregate scores increased by between 15\% and 30\%
across all topics.
Instructor observation found the students eager to participate in the
tricks. Students reported that the lessons increased their mastery of the material
and described it as ``fun,''
``entertaining,'' and ``interesting.''

\section{Contributions and Future Work}

The main contributions in this work are as follows:

\begin{itemize}

\item{We create a lesson plan that includes three easy-to-perform magic
    tricks based on three specific types of attacks}

\item{We provide an evaluation that speaks to the lessons' effectiveness}

\item{We demonstrate an improved ability to answer questions related to the
    attacks following our lesson}

\item{We discuss future plans to carry out larger, in-person sessions with
    undergraduate students to explore our lesson's efficacy in that
        setting}

\end{itemize}

\bibliographystyle{ACM-Reference-Format}
\bibliography{bibdata}

\newpage
\section{Context for Reviewers}
\subsection{Poster Layout}
The poster of two sections. The first section will present the operative
steps of each trick and the cybersecurity concept it teaches.  This section
will consist of graphical illustrations of each step that the presenter
will use to walk participants through how each trick works.  The goal is to
leave participants with the ability to perform at least one of the tricks
in a classroom setting.

The second section of the poster will cover the results of the live
sessions that were conducted over the summer.  The goal of this section is
to convey the effectiveness of the lessons taught using the tricks and give
participants an opportunity to ask questions about methods, results, or the
experience of teaching a live lesson in the form of a magic show.


\subsection{Handouts}

Participants will receive a ``cheat sheet'' describing the operative steps
of each trick.  The goal of this document is to allow instructors to
refresh their memory should they wish to use the tricks as part of their
own lessons.

\subsection{Presenter Background}

The presenter organized and conducted the live sessions that formed the
basis for this works evaluation.  They are skilled at both executing the
tricks in a live setting and explaining to participants how the tricks
work.  They also have a strong cybersecurity and computer science
background that they use to relate the tricks to their respective
cybersecurity topics.

\subsection{Participant Engagement}

Rather than standard discussion, the presenter will demonstrate the
tricks used in this work as part of an abbreviated presentation
to participants.
The goal is to show participants trick's impact on an audience so they can
gain an appreciation for how they help students relate to a topic.
Along with this presentation, the presenter will
cover the purpose of each trick and the results of the study conducted as
part of this work.


\end{document}
